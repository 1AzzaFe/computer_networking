\documentclass{article}
\usepackage{graphicx} % Required for inserting images


\title{Redes de Computadores}
\author{Asafe S. Meireles}
\date{December 2024}

\begin{document}


\section{Básico de redes}
Protocolos em roteadores determinam a origem e o destino de um pacote ou mais exemplos são protocolos executados no hardware de dois computadores conectados fisicamente controlam o fluxo de
bits no “cabo” entre as duas placas de interface de rede. 
Em resumo todo comunicação feita na internet que envalva duas ou mais entidades são controladas por redes. 

\section{Componetes da rede}
Vamos chamas de \textit{sistemas finais} os dispositivos que carregam programas de aplicações com browser, formalmentes chamaremos de \textit{hospedeiros} ou \textit{hosts}.
Temos clientes e servidores, clientes são os computadores ou celulares móveis, enquanto servidores são maquinas mais poderosas, como os servidores propiamentes ditos, distribuem sites web.

\section{Redes de acesso}
Temos agora os casos em que os \textit{hospedeiros} se conectam as roteadores também chamado roteador de borda.
\\\\
\textbf{Acesso remoto}
\\\\
\textbullet{ DSL, FTTH, Satélite, Discado, Cabo.}
\\\\
\textbf{Na residência: Ethernet e Wi-Fi}
\\\\
\textbf{Acesso sem fio em longa distância: 3G e LTE}\\\\
prete atenção em como cada uma das tecnológias funcioam, em que meio físico são utulizados.

\section{Comutação de pacotes}
Da origem até um destino, uma mensagem é fragentada, damos o nome desses fragmentos de \textit{pacotes}. 
Da origem até o destino os \textit{pacotes} percorrem enlaces e \textbf{computadores de pacotes}, no qual existem dois: \textbf{roteadores}e \textbf{comutadores de camada de enlace}.
L bits por enlace a uma taxa R bits/s o tempo para tranmitir o pacote é de:
$$ \varphi = \frac{L}{S}sec $$
\end{document}

